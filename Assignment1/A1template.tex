\documentclass[12pt]{article}
\usepackage{graphics}
\usepackage[top=1in,bottom=1in,left=1in,right=1in]{geometry}
\usepackage{alltt}
\usepackage{array}	
\usepackage{graphicx}
\usepackage{tabularx}
\usepackage{verbatim}
\usepackage{setspace}
\usepackage{listings}

\usepackage{amssymb,amsmath, amsthm}
\usepackage{zed-csp}
\usepackage[cc]{titlepic}

\title{\textbf{SOEN 331 Section S: Formal Methods\\for Software Engineering\\
 \\ 
Assignment 1}}
\author{Mohammad Ali Zahir - 40077619 & Marwa Khalid - 40155098}
\date{\today \\ & \ \\ Date of Submission: 9 October. 2022}
\begin{spacing}{1.5}
\begin{document}
\maketitle



\newpage
\begin{enumerate}

\item Let $B(x)$ denote the subject ``x is a bird'' and $W(x)$ denote the predicate ``x is white.'' Translate the following formal statements into English sentences and attach the corresponding categorical form to each:

\begin{enumerate}
\item $\forall x~(B(x) \rightarrow W(x))$


\item $\forall x~(B(x) \rightarrow \neg W(x))$ 


\item $\exists x~(B(x) \wedge W(x))$ 


\item $\exists x~(B(x) \wedge \neg W(x))$ 
\end{enumerate}

\noindent \underline{Solution}: \\

\begin{enumerate}
\item $\forall x~(B(x) \rightarrow W(x))$ : ``All birds are white.'': (A)
\item $\forall x~(B(x) \rightarrow \neg W(x))$ : ``All birds are non-white.'': (E)
\item $\exists x~(B(x) \wedge W(x))$ : ``Some birds are white.'': (I)
\item $\exists x~(B(x) \wedge \neg W(x))$ : ``There is an non-white bird.'': (O)
\end{enumerate}


\item Describe when the following predicate can be false: $\forall x \exists y P(x, y)$.\\

\noindent \underline{Solution}: The statement can be false when there is an $x$ such that $P(x, y)$ is false for every $y$.



\item Let $P(x, y)$ be the statement ``x asked y out to lunch'' where the domain is all students in class. Express each of the following quantifications in English:

\begin{enumerate}
\item $\exists y \forall x P(x, y)$.\\

\noindent \underline{Solution}: Recall that this reads \textbf{''There in an $y$ that makes $P(x, y)$ true for every $x$.''} There is a student in class who has been asked out to lunch by every student in class.\\

\item $\forall x \exists y P(x, y)$.\\

\noindent \underline{Solution}: Recall that this reads \textbf{``For every $x$ there is a $y$ for which $P(x, y)$ is true.''} Every student in class has asked out to lunch some (at least one) student in class.
\end{enumerate}


\end{enumerate}





\noindent \textbf{An important observation on functions is that we can view them as relations, and as such we can model a function as a set of pairs (tuples)}.

\begin{enumerate}

\item Consider the following relation:

\[ phone : Model \leftrightarrow Brand \]

\noindent where

\[
phone = \\
\{ \\
\hspace{10mm} iPhone7 \mapsto apple,\\
\hspace{10mm} iPhoneX \mapsto apple,\\
\hspace{10mm} galaxyS \mapsto samsung,\\
\hspace{10mm} galaxyA \mapsto samsung,\\
\hspace{10mm} galaxyJ \mapsto samsung,\\
\hspace{10mm} mate20 \mapsto huawei,\\
\hspace{10mm} p20 \mapsto huawei\\
\}
\]

\begin{enumerate}

\item What is the domain of the relation?\\

\noindent $\dom~phone = \{ iPhone7, iPhoneX, galaxyS, galaxyA, galaxyJ, mate20, p20 \}$.


\newpage


\item What is the range of the relation?\\

\noindent $\ran~phone = \{ apple, samsung, huawei \}$.


\ \\


\item What is the result of the expression  $\{ iPhone7, galaxyA \}  \lhd phone$ ?\\


\noindent Domain restriction selects pairs based on their first element. As a result,

\[ \{ iPhone7, galaxyA \}  \lhd phone = \{ iPhone7 \mapsto apple, galaxyA \mapsto samsung \} \]


\noindent Restriction operators are deployed to model database \textit{queries}.


\ \\


\item What is the result of the expression  $phone \rhd \{ apple, samsung \}$ ?\\


\noindent Range restriction selects pairs based on their second element. As a result,

\[ 
phone \rhd \{ apple, samsung \} = \\

\hspace{5mm} \{ \\
\hspace{10mm} iPhone7 \mapsto apple,\\
\hspace{10mm} iPhoneX \mapsto apple,\\
\hspace{10mm} galaxyS \mapsto samsung,\\
\hspace{10mm} galaxyA \mapsto samsung,\\
\hspace{10mm} galaxyJ \mapsto samsung\\
\hspace{5mm} \}
\]


\newpage

\noindent Consider the following Questions to be done in one sequence where we will make permanent modifications to the contents of $phone$:\\


\item What is the result of the expression  $\{ iPhone7, iPhoneX, galaxyA, mate20 \} \ndres phone$ ?\\


\noindent  Domain subtraction removes elements from the domain of the relation:\\


\[
\{ iPhone7, iPhoneX, galaxyA, mate20 \} \ndres phone = \\

\hspace{5mm} \{ \\
\hspace{10mm} galaxyS \mapsto samsung,\\
\hspace{10mm} galaxyJ \mapsto samsung,\\
\hspace{10mm} p20 \mapsto huawei\\
\hspace{5mm} \}

\]

\noindent Note that for a modification to phone, we need to write


\[ phone' = \{ iPhone7, iPhoneX, galaxyA,mate20  \} \ndres phone \]


\noindent which reads as: ``The new value of phone is assigned the value of the evaluation of
the expression on the right-hand-side.''



\item What is the result of the expression  $phone \nrres \{ huawei \}$ ?\\


\noindent  Range subtraction removes elements from the codomain of the relation:\\


\[
phone \nrres \{ huawei\} = \\

\hspace{5mm} \{ \\
\hspace{10mm} galaxyS \mapsto samsung,\\
\hspace{10mm} galaxyJ \mapsto samsung\\
\hspace{5mm} \}

\]


\noindent Assume now that we did

\[ phone' = phone \nrres \{ huawei\} \]


\noindent and as a result, the new value of $phone$ will be

\[
phone = \\

\hspace{5mm} \{ \\
\hspace{10mm} galaxyS \mapsto samsung,\\
\hspace{10mm} galaxyJ \mapsto samsung\\
\hspace{5mm} \}

\]


\ \\


\item What is the result of $phone \oplus \{ iPhoneXSMax \mapsto apple \}$? \\


\noindent Relational overriding can model database updates.

\[
phone \oplus \{ iPhoneXSMax \mapsto apple \} = \\
\hspace{5mm} \{\\
\hspace{10mm} iPhoneXSMax \mapsto apple, \\
\hspace{10mm} galaxyS \mapsto samsung, \\
\hspace{10mm} galaxyJ \mapsto samsung \\
\hspace{5mm} \}
\]

\noindent Note that for a modification to $phone$, we need to write

\[ phone' = phone \oplus \{ iPhoneXSMax \mapsto apple \} \]

\end{enumerate}



\item Consider the sets

\begin{itemize}
\item $Phone = \{ Samsung, Huawei, Apple, Sony, Motorola, HTC \}$, and
\item  $Favorite = \{ Sony, HTC \}$.
\end{itemize}

\noindent Answer the following questions:

\begin{enumerate}

\item How do we interpret the expression $Favorite : \mathbb{P} Phone$?

\item Is $\mathbb{P} Phone$ a legitimate type?

\item What is the nature of the variable in $Favorite : \mathbb{P} Phone$? (i.e. atomic or composite?)

\item Is $Apple \in \mathbb{P} Phone$?

\item Is $\{ Apple \} \in \mathbb{P} Phone$?

\item Is $\{ \{ \} \} \in \mathbb{P} Phone$?

\item Is $\{ \} \in \mathbb{P} Phone$?

\item If we define variable $Favorite : \mathbb{P} Phone$, is $\{ \}$ a legitimate value for variable $Favorite$?

\item Is $Favorite \in \mathbb{P} Phone$?

\item Is $Favorite \subset \mathbb{P} Phone$?

\end{enumerate}

\noindent \underline{Solution}:

\begin{enumerate}

\item How do we interpret the expression $Favorite : \mathbb{P} Phone$? \underline{Answer}: This is  interpreted as ``The variable $Favorite$ can assume any value supported by the powerset of $Phone$.

\item Is $\mathbb{P} Phone$ a legitimate type? \textbf{Yes.}

\item What is the nature of the variable in $Favorite : \mathbb{P} Phone$? \underline{Answer}: Variable $Favorite$ is a \textbf{set}.

\item Is $Apple \in \mathbb{P} Phone$? \textbf{No}.

\item Is $\{ Apple \} \in \mathbb{P} Phone$?  \textbf{Yes}.

\item Is $\{ \{ \} \} \in \mathbb{P} Phone$? \textbf{No}.

\item Is $\{ \} \in \mathbb{P} Phone$? \textbf{Yes}.

\item If we define variable $favorite : \mathbb{P} Phone$, is $\{ \}$ a legitimate value for variable $Favorite$? \textbf{Yes}.

\item Is $Favorite \in \mathbb{P} Phone$? \textbf{Yes}.

\item Is $Favorite \subset \mathbb{P} Phone$? \textbf{No}.

\end{enumerate}










\item Consider the following relation:

\[ laptops : Model \leftrightarrow Brand \]

\noindent where

\[
laptops = \\
\hspace{5mm} \{ \\
\hspace{10mm} legion5 \mapsto lenovo,\\
\hspace{10mm} macbookair \mapsto apple,\\
\hspace{10mm} xps15 \mapsto dell,\\
\hspace{10mm} spectre \mapsto hp,\\
\hspace{10mm} xps13 \mapsto dell,\\
\hspace{10mm} swift3 \mapsto acer,\\
\hspace{10mm} macbookpro \mapsto apple,\\
\hspace{10mm} dragonfly \mapsto hp,\\
\hspace{10mm} envyx360 \mapsto hp\\
\hspace{5mm} \}
\]

\begin{enumerate}

\item What is the domain and the range of the relation?\\

\noindent \underline{Answer}:

\begin{itemize}
\item The domain is defined as:  % \textbf{\underline{Marking:  0.5 pt}}
\[
\dom laptops = \\
\hspace{5mm} \{ \\
\hspace{10mm} legion5,\\
\hspace{10mm} macbookair,\\
\hspace{10mm} xps15,\\
\hspace{10mm} spectre,\\
\hspace{10mm} xps13,\\
\hspace{10mm} swift3,\\
\hspace{10mm} macbookpro,\\
\hspace{10mm} dragonfly,\\
\hspace{10mm} envyx360\\
\hspace{5mm} \}
\]


\item The range is defined as:  $\ran~laptops = \{ lenovo, apple, dell, hp, acer \}$.


\end{itemize}


\item What is the result of the expression

\[ \{ xps15, xps13, swift3, envyx360 \}  \lhd laptops \]

\noindent What is the meaning of operator $\lhd$ and where would you deploy such operator in the context of a database management system?

\newpage

\noindent \underline{Answer}:

\noindent The result is 

\[ \{ xps15, xps13, swift3, envyx360 \}  \lhd laptops = \\
\hspace{5mm} \{ \\
\hspace{10mm} xps15 \mapsto dell,\\
\hspace{10mm} xps13 \mapsto dell,\\
\hspace{10mm} swift3 \mapsto acer,\\
\hspace{10mm} envyx360 \mapsto hp\\
\hspace{5mm} \}
\]

\noindent Domain restriction selects pairs based on their first element. We deploy such operators to model database queries. 


\item What is the result of the expression

\[ laptops \rhd \{ lenovo, hp \} \]

\noindent What is the meaning of operator $\rhd$ and where would you deploy such operator in the context of a database management system?

\noindent \underline{Answer}:

\noindent The result is 


\[ laptops \rhd \{ lenovo, hp \} = \\
\hspace{5mm} \{ \\
\hspace{10mm} legion5 \mapsto lenovo,\\
\hspace{10mm} spectre \mapsto hp,\\
\hspace{10mm} dragonfly \mapsto hp,\\
\hspace{10mm} envyx360 \mapsto hp\\
\hspace{5mm} \}
\]

\noindent Range restriction selects pairs based on their second element. We deploy such operators to model database queries. 


\item What is the result of the expression

\[ \{ legion5, xps15, xps13, dragonfly \} \ndres laptops \]

\noindent What is the meaning of operator $\ndres$ and where would you deploy such operator in the context of a database management system?

\noindent \underline{Answer}:

\noindent \noindent The result is 

\[ \{ legion5, xps15, xps13, dragonfly \} \ndres laptops = \\
\hspace{5mm} \{ \\
\hspace{10mm} macbookair \mapsto apple,\\
\hspace{10mm} spectre \mapsto hp,\\
\hspace{10mm} swift3 \mapsto acer,\\
\hspace{10mm} macbookpro \mapsto apple,\\
\hspace{10mm} envyx360 \mapsto hp\\
\hspace{5mm} \}
\]

\noindent  Domain subtraction removes elements from the domain of the relation. We deploy such operation to model deletion of records. 


\item What is the result of the expression

\[ laptops \nrres \{ apple, dell, hp \} \]

\noindent What is the meaning of operator $\nrres$ and where would you deploy such operator in the context of a database management system?

\newpage

\noindent \underline{Answer}:


\noindent \noindent The result is 

\[
laptops \nrres \{ apple, dell, hp \} = \\
\hspace{5mm} \{ \\
\hspace{10mm} legion5 \mapsto lenovo,\\
\hspace{10mm} swift3 \mapsto acer\\
\hspace{5mm} \}
\]
 
 
\noindent  Range subtraction removes elements from the codomain of the relation. We deploy such operation to model database updates (deletion of records). % \textbf{\underline{Marking:  1.5 pts}}



\item Consider the following expression

\[ laptops \oplus \{ ideapad \mapsto lenovo \} \]


\begin{enumerate}
\item What is the result of the expression?

\item What is the meaning of operator $\oplus$ and where would you deploy such operator in the context of a database management system?

\item Does the result of the expression have a permanent effect on the database (relation)? If not, describe in detail how would you ensure a permanent effect. 

\end{enumerate}

\newpage

\noindent \underline{Answer}:


\begin{enumerate}
\item  The result is 

\[ laptops \oplus \{ ideapad \mapsto lenovo \} = \\
\hspace{5mm} \{ \\
\hspace{10mm} ideapad \mapsto lenovo,\\
\hspace{10mm} legion5 \mapsto lenovo,\\
\hspace{10mm} macbookair \mapsto apple,\\
\hspace{10mm} xps15 \mapsto dell,\\
\hspace{10mm} spectre \mapsto hp,\\
\hspace{10mm} xps13 \mapsto dell,\\
\hspace{10mm} swift3 \mapsto acer,\\
\hspace{10mm} macbookpro \mapsto apple,\\
\hspace{10mm} dragonfly \mapsto hp,\\
\hspace{10mm} envyx360 \mapsto hp\\
\hspace{5mm} \}
\]

\item Relational overriding can model database updates (addition of records).


\item The expression does not have a permanent effect on the database (relation). To ensure a permanent effect on the relation, we need to define an assignment statement

\[ laptops' = laptops \oplus \{ ideapad \mapsto lenovo \} \]

\noindent which reads ``The value of variable (relation) $laptops$ is assigned the result of the expression on the right-hand-side of the assignment statement.''

\end{enumerate}
\end{enumerate}














\end{enumerate}

\end{spacing}

\end{document}
