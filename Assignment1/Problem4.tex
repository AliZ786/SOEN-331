\section{Problem 4: Relational calculus 1 (15 pts)}

\subsection{Description:}

\noindent Consider a system that associates active flights to airlines. The requirements of the system
are as follows:

\begin{itemize}
  \item Flights are unique.
  \item Each flight is associated to a single airline, e.g. AA333 is an American Airlines flight.
  \item The system can support new flights to be associated to an existing airline, or existing flights to be deleted.
  \item Airlines can have several active flights at any point in time.
\end{itemize}

\noindent We introduce types \emph{Flight} and \emph{Airline}. The model of the system is captured by variable
map, as shown below:\\

\[
map = \\
\hspace{5mm} \{ \\
\hspace{10mm} AAA333 \mapsto \emph{American Airlines},\\
\hspace{10mm} AY29 \mapsto \emph{Finnair},\\
\hspace{10mm} TS261 \mapsto \emph{Air Transat},\\
\hspace{10mm} TS765 \mapsto \emph{Air Transat}\\
\hspace{5mm} \}
\]

\begin{enumerate}
  \item Is \emph{map} a binary relation? Explain.\\
  \noindent\underline{Solution:}\\ Since this relation relates the the elemments of one set (flight), which is the domain, with the elements of another set {airplane} which is the codomain. \textbf{The relation map is hence considered a binary relation}.
  \item Is \emph{map} a function? Explain and if Yes, determine the type of the function.\\
  \noindent\underline{Solution:}\\ We could say that this is a function since all elements of the domain are unique and are linked to exactly one member of the codomain.\\
  In terms of what kind of function this is, we cannot say this is one-to-one function since both TS261 and TS765 are both linked to the Air Transat airline. Since this confirms that the fucntion is not one-to-one,
  we can also confirm that this is not a bijective function since for a function to be bijective, it has to be onto and one-to-one. \textbf{Hence, we can say that this function is onto, since all elements of the codomain (airline), is mapped by atleast one member of the domain (flight)}.
  \item Define the precondition for operation \emph{add}, that adds a new flight-airline pair.\\
  \noindent\underline{Solution:}\\ Intially we would need to check if the ordered pair exists in maps. To check this, we would need to see that the flight is not contained in the database.\\
  If that doesn't exist, then we can add the new ordered pair. The translated statement for the add would be:\\
  \textbf{\emph{flight $\notin$ dom map}}\\

  \textbf{For Questions 4 and 5 assume the presence of the above precondition:}\\
  \item Provide two alternative definitions for the core functionality of operation \emph{add}.\\
  \noindent\underline{Solution:}\\ To add the operation, we can either do a union or we can do the insertion operator. The statements needed for those would look like the following:\
  \begin{itemize}
    \item \emph{map'} = \emph{map= $\cup\ \{flight \mapsto airline \}}$
    \item \emph{map'} = \emph{map= $\oplus\ \{flight \mapsto airline \}}$
  
  \end{itemize}
  \textbf{For Questions 6 and 7 assume that the above precondition is removed:}\\
  \item What would be the result of calling operation \emph{add} with \emph{flight}? = \emph{TS765}, and \emph{airline}? = \emph{American Airlines}?\\
  \noindent\underline{Solution:}\\ Since the flight TS765 already exists in the domain of maps, \textbf{this operation will give us a failure}.
  \item What would be the result of calling \emph{add} with\
  \[
  \hspace{10mm} \emph{flight}? = AA333,\\
  \hspace{10mm} \emph{airline}? = Air Canada,
  \]
  \noindent\underline{Solution:}\\ Since this flight does not exist in the domain of maps, we would be able to add this ordered pair.\\
  We would the use the insertion operator, which would look like:\
  \[
    \emph{map'} = map \oplus\{AA333 \mapsto Air Canada\}  = \
    \hspace{5mm} \{ \\
    \hspace{10mm} AA333 \mapsto \emph{Air Canada},\\
    \hspace{10mm} AAA333 \mapsto \emph{American Airlines},\\
    \hspace{10mm} AY29 \mapsto \emph{Finnair},\\
    \hspace{10mm} TS261 \mapsto \emph{Air Transat},\\
    \hspace{10mm} TS765 \mapsto \emph{Air Transat}\\
    \hspace{5mm} \}
      \]
  
  \item Under what conditions, if any, can \emph{set union} serve as a mechanism to successfully add a new record into the database table? What error could possibly occur?\\
  \noindent\underline{Solution:}\\ Set Union is used to add a new record. It would first check if the record exists by checking if the domain exists in the database already.\\
  If such an element does not exist in the database, we add the new ordered pair. However, since this is a union operation, it will still add the ordered pair even if the element does exists, which would mean we would have duplicates in our databse.\\
  This will make it that we will have two elements in the same domain (which is flight), which violates the condition that all flights are unique.
  
  \item Provide a definition for the core functionality of operation \textbf{delete} that erases a flight from \emph{map}, given the flight number.\\
  \noindent\underline{Solution:}\\Intially we would need to check if the ordered pair exists in maps. We would need to see that the flight is contained in the databse.\\
  If that does exist, then we can delete the existing ordered pair. The translated statement for the delete would be:\\
  \textbf{\emph{flight $\in$ dom map}}\\

  
  


\end{enumerate}