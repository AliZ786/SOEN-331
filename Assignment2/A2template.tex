\documentclass[12pt]{article}
\usepackage[top=1in,bottom=1in,left=1in,right=1in]{geometry}
\usepackage{alltt}
\usepackage{array}	
\usepackage{graphicx}
\usepackage{tabularx}
\usepackage{verbatim}
\usepackage{setspace}
\usepackage{listings}
\usepackage{amssymb,amsmath, amsthm}
\usepackage{hyperref}
\usepackage{oz}
\usepackage[cc]{titlepic}
\usepackage{fancyvrb}


\title{Concordia University\\
Department of Computer Science and Software Engineering\\
\textbf{SOEN 331:\\Formal Methods for Software Engineering}\\
\ \\
\textbf{Exercise in Z}}
\author{\textbf{Dr. Constantinos Constantinides, P.Eng.}\\
\ \\}
\date{\today}

\begin{spacing}{1.5}

\begin{document}
\maketitle

\newpage

\section*{Temperature monitoring system with the Z specification}

Consider a system called 'TempMonitor' that keeps a number of sensors, where each sensor is deployed in a separate location in order to read the location's temperature. Before the system is deployed, all locations are marked on a map, and each location will be addressed by a sensor. The formal specification of the system introduces the following three types:

\[ SENSOR\_TYPE,  LOCATION\_TYPE, TEMPERATURE\_TYPE  \]

\noindent We also introduce an enumerated type $MESSAGE$ which will assume values that correspond to success and error messages.\\

\noindent Provide a formal specification in Z, with the following operations:

\begin{itemize}
	\item \texttt{DeploySensorOK}:  Places a new sensor to a unique location. You may assume that some (default) temperature is also passed as an argument.
	\item \texttt{ReadTemperatureOK}: Obtain the temperature reading from a sensor, given the sensor's location.
\end{itemize}

\noindent Provide appropriate success and error schemata to be combined with the definitions above to produce robust specifications for the following interface:

\begin{itemize}
	\item \texttt{DeploySensor},
	\item \texttt{ReadTemperature}.
\end{itemize}

\newpage

\noindent \underline{Solution}:

\begin{schema}{TempMonitor}
deployed'~:~\mathbb{P}~SENSOR\_TYPE\\
map : SENSOR\_TYPE \nrightarrow LOCATION\_TYPE \texttt{~~~~~--partial bijective}\\
read : SENSOR\_TYPE  \nrightarrow TEMPERATURE\_TYPE\\
\where
deployed = \dom map\\
deployed = \dom read
\end{schema}

\begin{schema}{DeploySensorOK}
\Delta TempMonitor\\
sensor? : SENSOR\_TYPE\\
location? : LOCATION\_TYPE\\
temperature? : TEMPERATURE\_TYPE
\where
sensor? \notin deployed\\
location? \notin \ran map\\
deployed' = deployed \cup \{ sensor? \}\\
map' = map \cup \{ sensor? \mapsto location? \}\\
read' = read \cup \{ sensor? \mapsto temperature? \}
\end{schema}


\begin{schema}{ReadTemperatureOK}
\Xi TempMonitor\\
location? : LOCATION\_TYPE\\
temperature! : TEMPERATURE\_TYPE
\where
location? \in \ran map\\
temperature! = read(map^{-1}(location?))\\
\end{schema}

\begin{schema}{Success}
\Xi TempMonitor\\
response! : MESSAGE
\where
response!~=~'ok'\\
\end{schema}



\begin{schema}{SensorAlreadyDeployed}
\Xi TempMonitor\\
sensor? : SENSOR\_TYPE\\
response! : Message
\ST
sensor? \in deployed\\
response!~=~'Sensor~deployed'
\end{schema}


\begin{schema}{LocationAlreadyCovered}
\Xi TempMonitor\\
location? : LOCATION\_TYPE\\
response! : Message
\ST
location? \in \ran map\\
response!~=~'Location~already~covered'
\end{schema}


\begin{schema}{LocationUnknown}
\Xi TempMonitor\\
location? : LOCATION\_TYPE\\
response! : Message
\ST
location? \notin \ran map\\
response!~=~'Location~not~covered'
\end{schema}

\[ DeploySensor~\hat{=}~\\
~~~(DeploySensorOK \wedge Success) \oplus (SensorAlreadyDeployed \vee LocationAlreadyCovered) \]



\[ ReadTemperature~\hat{=}~(ReadTemperatureOK \wedge Success) \oplus LocationUnknown \]


\end{spacing}

\end{document}
