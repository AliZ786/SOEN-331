\documentclass[12pt]{article}
\usepackage[top=1in,bottom=1in,left=1in,right=1in]{geometry}
\usepackage{alltt}
\usepackage{array}	
\usepackage{graphicx}
\usepackage{tabularx}
\usepackage{verbatim}
\usepackage{setspace}
\usepackage{listings}
\usepackage{amssymb,amsmath, amsthm}
\usepackage{hyperref}
\usepackage{oz}
\usepackage[cc]{titlepic}
\usepackage{fancyvrb}


\title{\textbf{Concordia University
Department of Computer Science and Software
Engineering} \\ \ \\SOEN 331 Section S: Formal Methods\\for Software Engineering\\
 \\ 
Assignment 2}
\author{Mohammad Ali Zahir - 40077619 & Marwa Khalid - 40155098}
\date{October 17, 2022 \\ & \ \\ Date of Submission: October 31, 2022}
\begin{spacing}{1.5}
\begin{document}
\maketitle

\newpage
\tableofcontents
\newpage

\section{System Requirements}
\noindent Consider a system such as \emph{flightradar24.com}. A flight is associated with \textbf{a flight number}
(such as UA79), a specific code that an airline assigns to a particular flight in its network,
and \textbf{route} which is a source-destination city pair such as \textit{(NY ,Tokyo)}. For example, the
United Airlines flight from New to Tokyo is tracked by the system as ${UA79 \mapsto  (NY ,Tokyo)$}.
The formal specification of the system introduces the following three types:
\\

\[
  \hspace{5mm} \emph{FLIGHT\_NUMBER},\\
  \hspace{5mm} \emph{ROUTE}, \\
  \hspace{5mm} \emph{CITY}
\]

\\
\noindent where \\
\noindent ${ROUTE: CITY \times CITY $} \\

\noindent Flight numbers are unique, and there are possibly several flights that cover the same route.
For example, there are possibly several flights from New York to Tokyo. The system must
keep track of all active flights. Formally, let us have the following variables:

\begin{enumerate}
\item \emph{active}: holds all active flight numbers.
\item \emph{map}: holds a collection of active flight-route pairs.
\end{enumerate}

\section{Your Assignment}

\begin{enumerate}
\item (2 pts) Provide a declaration for variable \emph{active}. \\
\noindent \underline{Solution:}\\
\item (3 pts) What kind of collection is variable \emph{map}. \\
\noindent \underline{Solution:}\\
\item (10 pts) Is variable map a function and if so, comment on whether it is a total or
partial function, as well as on the properties of injectivity, surjectivity and bijectivity? \\
\noindent \underline{Solution:}\\
\item (10 pts) Provide a formal specification of the state of the system in terms of a \textbf{Z
specification schema}. \\
\noindent \underline{Solution:}\\
\item (15 pts) Provide a schema for operation \emph{RegisterFlightOK} that adds a flight to the
tracker. With the aid of success and error schema(s), provide a definition for operation
\emph{RegisterFlight} that the system will place in its exposed interface. \\
\noindent \underline{Solution:}\\
\item (15 pts) Provide a schema for operation \emph{GetRouteOK} that returns the route given its
flight. With the aid of success and error schema(s), provide a definition for operation
\emph{GetRoute} that the system will place in its exposed interface. \\
\noindent \underline{Solution:}\\
\item Provide a schema for operation \emph{GetFlightOK} that returns any and all active
flights given a route. With the aid of success and error schema(s), provide a definition
for operation \emph{GetFlight} that the system will place in its exposed interface. \\
\noindent \underline{Solution:}\\

\end{enumerate}



\end{spacing}

\end{document}
