\documentclass[12pt]{article}
\usepackage[top=1in,bottom=1in,left=1in,right=1in]{geometry}
\usepackage{alltt}
\usepackage{array}	
\usepackage{graphicx}
\usepackage{tabularx}
\usepackage{verbatim}
\usepackage{setspace}
\usepackage{listings}
\usepackage{amssymb,amsmath, amsthm}
\usepackage{hyperref}
\usepackage{oz}
\usepackage[cc]{titlepic}
\usepackage{fancyvrb}


\title{\textbf{Concordia University
Department of Computer Science and Software
Engineering} \\ \ \\SOEN 331 Section S: Formal Methods\\for Software Engineering\\
 \\ 
Assignment 4}
\author{Mohammad Ali Zahir - 40077619 & Marwa Khalid - 40155098}
\date{November 23, 2022 \\ & \ \\ Date of Submission: December 2, 2022}
\begin{spacing}{1.5}
\begin{document}
\maketitle

\newpage
\tableofcontents
\newpage

\section{Our assignment}
\begin{enumerate}
\item (10 pts) Find a logically equivalent formula for $\phi \ W \ \psi$ and provide a short reasoning
to support your answer. Represent this equivalence between the two expressions with
the appropriate logical connective, and support your reasoning.\\\
\\
\noindent \underline{Solution}:\\ The logically equivalent formula for this would be:\\
\( (\phi \ \textbf{W} \ \psi) \equiv \ $(\phi \ \textbf{U} \ \psi$)\ \vee \ $\square(\psi)$ \\
\noindent These are equivalent, because the principle of the strong until operator \textbf{U}. Since the $\psi$ is never guaranteed to be true, we would need to add an extra or statement because if the statement $\psi$ becomes true it would mean that $\phi$ can't be true. This means that $\phi$ would be true until a certain condition ($\psi$ is true) is met. \\

\item (10 pts) Find a logically equivalent formula for $\phi \ U \ \psi$ , and provide a short reasoning
to support your answer. Represent this equivalence between the two expressions with the appropriate logical connective, and support your reasoning.\\\
\\
\noindent \underline{Solution}:\\ The logically equivalent formula for this would be: \\
$(\phi \ \textbf{U} \ \psi$) \equiv \  $(\phi \ \textbf{W} \ \psi) \ \wedge \ $\bigcirc \lozenge(\psi)$ \\
\noindent These are equivalent, because the principle of the strong until operator U. We know that this $\psi$ will eventually become true. We then now that we can use the weak until clause with an eventually operator, because the only way that $\phi$ is not true is when the $\psi$ is not true. We add the and operator, because we want to insure that the next one we actually get $\psi$ as being true, because this may never happen. \\

\item (10 pts) Find a logically equivalent formula for $\phi \ R \ \psi$ in terms of W , and provide a short reasoning
to support your answer. Represent this equivalence between the two expressions with the appropriate logical connective, and support your reasoning.\\\
\\
\noindent \underline{Solution}:\\ The logically equivalent formula for this would be: \\
$(\phi \ \textbf{R} \ \psi$) \equiv \ $(\phi \ \textbf{W} \ \psi) \ \wedge \ $\lozenge(\psi)$ \\
\noindent These are equivalent as the weak until will make the operation hold until something is triggered. We need the second part of the equation to guarantee that $\psi$ will eventually has to be true, because if it is not then, this will never hold. Hence the and statement. \\
\\
\noindent This paragraph refers to Questions 4 - 5: Consider a railroad with a single rail and a
road level-crossing. We introduce the following propositions that represent events:
\begin{itemize}
\item[] a : A train is approaching.
\item[] b : The barrier is down
\item[] c : A train is crossing
\item[] l: A light is blinking
\end{itemize}
\item (15 pts) Express each of the following requirements formally. For each one, proceed to
find a logically equivalent formula that captures the safety property of the system (i.e.
in terms of “something bad never happens”):
\begin{itemize}
\item[(a)] (5 pts) When a train is crossing, the barrier must be down. \\
\noindent \underline{Solution}:\\ 
\square $(c \rightarrow \square b)$
\item[(b)] (5 pts) If a train is approaching or crossing, then the light must be blinking.
\noindent \underline{Solution}:\\
\square $(a \vee c \rightarrow \square l)$
\item[(c)] (5 pts) If the barrier is up and the light is off, then no train is coming or crossing.
\noindent \underline{Solution}:\\
\square $(\lnot(b \wedge l) \rightarrow \lnot \square(a \vee c)) $
\end{itemize}
\item (10 pts) Express each of the following requirements formally in terms of the liveness
property (i.e. in terms of “something good eventually happens”):
\begin{itemize}
\item[(a)] (5 pts) When a train is approaching, it will eventually cross.. \\
\noindent \underline{Solution}:\\
\square $(a \rightarrow \lozenge c) $
\item[(b)] (5 pts) When a train is approaching and no train is crossing, then the barrier will
eventually go down before the train crosses. \\
\noindent \underline{Solution}:\\
\square $((a \wedge \lnot c) \rightarrow \lozenge(\lnot b \ \textbf{U} \ c)) $
\end{itemize}
\item (45 pts) The behavior of a program is expressed by the following temporal formula:\\

\begin{figure}[h!]
\centering
\includegraphics[width=0.9\textwidth, height = 15cm, keepaspectratio]{images/Q6.png}
\end{figure}
\begin{itemize}
\item[(a)](20 pts) Visualize all models of behavior. \\
\noindent \underline{Solution}:\\
\begin{figure}[h!]
\centering
\includegraphics[width=0.9\textwidth, height = 15cm, keepaspectratio]{images/SOEN331_A4_Q6a.png}
\end{figure}
\begin{itemize}
\item[(b)](10 pts) Is the set of requirements satisfiable in all models of behavior? Explain
why or why not. \\
\noindent \underline{Solution}:\\ This set of requirements is not satistfiable in all behaviors. The first timeline, the top one terminates fine, hence it has no problems. The second one however has the operation $ \textbf{R} $ which means strong release. The specific condition that is using the strong release, would be the:
$\psi \wedge \tau \rightarrow \sigma \ \textbf{R} \ \pi $ \\
Hence at i =3, is when our strong release should have ended, but at i =4, we see an extra $ \pi $ which is not supposed to be there. \textbf{Hence this is not valid for all requirements}.  \\

\item[(c)] (10 pts) In the case where the set of requirements is not satisfiable, what modification(
s) to the requirements would you make (you may temporarily assume the
role of a stakeholder) in order to achieve satisfiability. \\
\noindent \underline{Solution}:\\
\item[(d)] (5 pts) Having resolved any possible conflicts in requirements, specify conditions
(models of behavior), if any exist, under which the program can terminate. If
none exist, please indicate so. \\
\noindent \underline{Solution}:\\
\end{itemize}


\end{enumerate}





\end{spacing}

\end{document}