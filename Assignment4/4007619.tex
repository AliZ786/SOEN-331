\documentclass[12pt]{article}
\usepackage[top=1in,bottom=1in,left=1in,right=1in]{geometry}
\usepackage{alltt}
\usepackage{array}	
\usepackage{graphicx}
\usepackage{tabularx}
\usepackage{verbatim}
\usepackage{setspace}
\usepackage{listings}
\usepackage{amssymb,amsmath, amsthm}
\usepackage{hyperref}
\usepackage{oz}
\usepackage[cc]{titlepic}
\usepackage{fancyvrb}


\title{\textbf{Concordia University
Department of Computer Science and Software
Engineering} \\ \ \\SOEN 331 Section S: Formal Methods\\for Software Engineering\\
 \\ 
Assignment 4}
\author{Mohammad Ali Zahir - 40077619 & Marwa Khalid - 40155098}
\date{November 23, 2022 \\ & \ \\ Date of Submission: December 2, 2022}
\begin{spacing}{1.5}
\begin{document}
\maketitle

\newpage
\tableofcontents
\newpage

\section{Our assignment}
\begin{enumerate}
\item Find a logically equivalent formula for $\phi \ W \ \psi$ and provide a short reasoning
to support your answer. Represent this equivalence between the two expressions with
the appropriate logical connective, and support your reasoning.\\\
\\
\noindent \underline{Solution}:\\ The logically equivalent formula would for this would be:\\
\( (\phi \ W \ \psi) \equiv \ $(\phi \ \oplus \ \psi$) \\
\noindent These are equivalent because  $\phi \ W \ \psi$ means that until the value for $\psi$ is false, $\phi$ is gonna return true. This is only acheivable with the
\textbf{XOR} operation since whenever the value for $\psi$ is 0, value for $\phi$ is equal to 1. and vice versa. 
\end{enumerate}





\end{spacing}

\end{document}